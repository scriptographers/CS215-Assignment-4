\documentclass[11pt, fleqn]{article}

\usepackage{amsmath}
\usepackage{amssymb}
\usepackage{amsthm}
\usepackage{mathtools}
\usepackage{hyperref}
\usepackage{ulem}
\usepackage{enumitem}
\usepackage[left=0.75in, right=0.75in, bottom=0.75in]{geometry}
% \usepackage{float}
\usepackage{floatrow}
\usepackage{graphicx}
\usepackage[export]{adjustbox}

\usepackage{sectsty}
\sectionfont{\centering}

\usepackage[perpage]{footmisc}

\usepackage{fancyhdr}
\pagestyle{fancy}
\fancyhf{}
\lhead{190100044 \& 190100055}
\rhead{CS 215: Assignment 4}
\renewcommand{\footrulewidth}{1.0pt}
\cfoot{Page \thepage}

\setlength{\parindent}{0em}
\renewcommand{\arraystretch}{2}%

\title{Assignment 4: CS 215}
\author{
\begin{tabular}{|c|c|}
     \hline
     Devansh Jain & Harshit Varma \\
     \hline
     190100044 & 190100055 \\
     \hline
\end{tabular}
}
\date{\today}

\begin{document}

\maketitle
\tableofcontents
\thispagestyle{empty}
\setcounter{page}{0}

\renewcommand{\arraystretch}{1}

\newpage
\section*{Question 1}
\addcontentsline{toc}{section}{Question 1}
\setcounter{equation}{0}
\setcounter{figure}{0}

\subsection*{(a)}
$X$ can take all values inside the unit square of area $4$ with equal probability.\\
Thus, the probability that $X$ takes values inside an arbitrary region of area $A$ (lying inside the unit square) is proportional to $A$.\\
$$
    P(X \in R) = k\cdot area(R)
$$
As the probability that $X$ lies inside the unit square is 1, this gives $k = \frac{1}{4}$.\\
A unit circle has an area of $\pi$, thus the probability that $X$ lies inside a unit circle is $\boxed{\frac{\pi}{4}}$

\subsection*{(b)}
Estimation of $\pi$ using $X$:\\
Generate $n$ samples of the form $(x_1, x_2)$ where  $x_1, x_2 \thicksim U(-1, 1)$.\\
Find the number of such points which satisfy $x_1^2 + x_2^2 \le 1$, let this number be $n_u$\\
Thus, $\frac{n_u}{n} \approx \frac{\pi}{4}$, an estimate of $\pi$ will be $\frac{4n_u}{n}$\\
More formally, let $I_{||X||\le 1}(X)$ be the indicator function that yields 1 if $||X|| = \sqrt{X_1^2 + X_2^2}\le 1$, and is 0 otherwise.\\
Then, $n_u = \sum_{i}^n I_{||x_i||\le 1}(x_i)$, thus an estimate for $\pi$ will be $\frac{\sum_{i}^n I_{||x_i||\le 1}(x_i)}{n}$

\subsection*{(c)}
Estimates of $\pi$ obtained for $N = 10, 10^2, 10^3, 10^4, 10^5, 10^6, 10^7, 10^8$ respectively
\begin{verbatim}
    N = 10, pi = 2.0000 
    N = 100, pi = 3.2400 
    N = 1000, pi = 3.2080 
    N = 10000, pi = 3.1316 
    N = 100000, pi = 3.1425 
    N = 1000000, pi = 3.1419 
    N = 10000000, pi = 3.1414 
    N = 100000000, pi = 3.1416 
\end{verbatim}
Our code handles $N = 10^9$ by computing $n_u$ using 10 $10^8$ (Similarly $\frac{n}{10^8}$ iterations for a general $n$, assuming it's a power of 10) sized arrays, by iterating 10 times, to ensure memory remains in check, although this takes quite a bit of time to execute. We get the output for $N = 10^9$ as \texttt{pi = 3.1416}
\begin{verbatim}
    n_fixed = single(10^8);
    n_large = single(10^9);
    n_iters = single(n_large/n_fixed);
    
    n_u = single(0);
    for i = 1:n_iters
        X1 = single(2*rand(n_fixed, 1)-1);
        X2 = single(2*rand(n_fixed, 1)-1);
        n_u = n_u + sum((X1.^2 + X2.^2) <= 1);
    end
\end{verbatim}

\subsection*{(d)}
$n_u = \sum_{i}^n I_{||x_i||\le 1}(x_i)$, also $I_{||x_i||\le 1}(x_i)$ is a Bernoulli random variable with the parameter $p = \frac{\pi}{4}$.\\
Thus, $n_u$ is the sum of $n$ Bernoulli RVs, i.e $n_u$ is binomially distributed.\\
Let $\hat \pi = \frac{4n_u}{n}$ be the estimate of $\pi$.\\
We need the minimum $n$ such that $ P(\pi - 0.01 \le \hat\pi \le \pi + 0.01) = 0.95$\\
Let $a = \pi - 0.01, b = \pi + 0.01$, thus, we need $P(a \le \frac{4n_u}{n} \le b) = P(\frac{na}{4} \le n_u \le \frac{nb}{4}) = 0.95$\\
We know that a binomial RV can be approximated by a Gaussian when $n$ is large (Central Limit Theorem).\\
(TODO: normalize $n_u$ to standard normal and then see the table for 95\% confidence interval)

\subsection*{Instructions for running the code:}
\begin{enumerate}[itemsep=-1ex]
    \item Unzip and \texttt{cd} to \texttt{code}, under this find the file named \texttt{q1.m}
    \item On running, it will print the estimates of $\pi$ for the given values of $n$, and it will also print the estimate for $\pi$ for $n = 10^9$ (This may take some time)
\end{enumerate}





\newpage
\section*{Question 2}
\addcontentsline{toc}{section}{Question 2}
\setcounter{equation}{0}
\setcounter{figure}{0}

\newpage
\section*{Question 3}
\addcontentsline{toc}{section}{Question 3}
\setcounter{equation}{0}
\setcounter{figure}{0}

\newpage
\section*{Question 4}
\addcontentsline{toc}{section}{Question 4}
\setcounter{equation}{0}
\setcounter{figure}{0}

\newpage
\section*{Question 5}
\addcontentsline{toc}{section}{Question 5}
\setcounter{equation}{0}
\setcounter{figure}{0}

\newpage
\section*{Question 6}
\addcontentsline{toc}{section}{Question 6}
\setcounter{equation}{0}
\setcounter{figure}{0}


\end{document}